\documentclass[11pt,a4paper]{article}

% font
\renewcommand{\familydefault}{\sfdefault}


\usepackage[margin=0.8in]{geometry}
\usepackage[utf8]{inputenc}
\usepackage{amsmath}
\usepackage{amsfonts}
\usepackage{amssymb}

\usepackage[hidelinks]{hyperref}
\usepackage{float}

\usepackage{lipsum}% http://ctan.org/pkg/lipsum

%% Bibliography/references packages
\usepackage[comma]{natbib}
%%\bibliographystyle{agsm}
\bibliographystyle{dcu}

%% https://en.wikibooks.org/wiki/LaTeX/List_Structures
\usepackage{scrextend}

% tables, row colour
\usepackage{tabularx,colortbl}
% For vertical centering text in X column
\renewcommand\tabularxcolumn[1]{m{#1}}

% https://tex.stackexchange.com/questions/22751/how-to-force-table-caption-on-top
%\usepackage[tableposition=top]{caption}
\usepackage{float}
\floatstyle{plaintop}
\restylefloat{table}

% https://en.wikibooks.org/wiki/LaTeX/List_Structures
\usepackage{enumitem}

% https://jansoehlke.com/2010/06/strikethrough-in-latex/
\usepackage{ulem}


%% Report variables
\newcommand{\scname}{BEN-1816}
\newcommand{\dlatestv}{1.00}

\definecolor{babyblueeyes}{rgb}{0.63, 0.79, 0.95}
\definecolor{ballblue}{rgb}{0.13, 0.67, 0.8}
\definecolor{beaublue}{rgb}{0.74, 0.83, 0.9}

\usepackage{array,booktabs,arydshln,xcolor}
\usepackage{xcolor}% http://ctan.org/pkg/xcolor
\usepackage{fancyhdr}% http://ctan.org/pkg/fancyhdr
\fancypagestyle{main}{%
	\renewcommand{\headrulewidth}{2pt}
	\renewcommand{\headrule}{\hbox to\headwidth{%
		\color{babyblueeyes}\leaders\hrule height \headrulewidth\hfill}}
	\renewcommand{\footrulewidth}{2pt}
	\renewcommand{\footrule}{\hbox to\headwidth{%
		\color{babyblueeyes}\leaders\hrule height \headrulewidth\hfill}}
	
	%\fancyhf{}
	%\fancyhead[LE]{\textbf{\leftmark}}
	%\fancyhead[RE]{\textbf{\scname{}}}
	%\fancyhead[LO]{\textbf{\scname{}}}
	%\fancyhead[RO]{\textbf{\rightmark}}

	%\fancyfoot[LE]{\textbf{\thepage}}
	%\fancyfoot[RE]{\textbf{\scname{} Configuration Guide}}
	%\fancyfoot[LO]{\textbf{\scname{} Configuration Guide}}
	%\fancyfoot[RO]{\textbf{\thepage}}
}


%% Make bibliography show in table of contents
%% https://tex.stackexchange.com/questions/8458/making-the-bibliography-appear-in-the-table-of-contents
\usepackage[nottoc,numbib]{tocbibind}
%% ^^^ overwrites \bibname, so set it back
\renewcommand{\bibname}{References}

\RequirePackage{filecontents}
\begin{filecontents}{prco304_h4.bib}
@inproceedings{safety_fpga,
  title={FPGAs in critical hardware/software systems},
  author={Hilton, Adrian J and Townson, Gemma and Hall, Jon G},
  booktitle={Proceedings of the 2003 ACM/SIGDA eleventh international symposium on Field programmable gate arrays},
  pages={244--244},
  year={2003},
  organization={ACM}
}
@online{defstan_0056,
	title={Safety Management Requirements for Defence Systems},
	url={https://segoldmine.ppi-int.com/content/standard-def-stan-00-56-safety-management-requirements-defence-systems},
	year={2007}
}
@article{defstan_0056_2,
  title={Safety-critical systems, formal methods and standards},
  author={Bowen, Jonathan and Stavridou, Victoria},
  journal={Software Engineering Journal},
  volume={8},
  number={4},
  pages={189--209},
  year={1993},
  publisher={IET}
}
@article{iec61508,
  title={Introduction to IEC 61508},
  author={Bell, Ron},
  booktitle={Proceedings of the 10th Australian workshop on Safety critical systems and software-Volume 55},
  pages={3--12},
  year={2006},
  organization={Australian Computer Society, Inc.}
}

@misc{scarabhardware,
title={MiniSpartan6+}, 
journal={{Scarab Hardware}},
url={https://www.scarabhardware.com/minispartan6/},
year=2014
}
@misc{arty,
title={Arty Artix-7 FPGA Development Board}, 
journal={{Avnet}},
url={https://uk.rs-online.com/web/p/programmable-logic-development-kits/1346478/},
year=2015
}
@misc{arndt2002algorithms,
  title={Algorithms For Programmers},
  author={Arndt, J{\"o}rg},
  year = 2002
}
@book{hdl,
title={HDL Programming Fundamentals: VHDL and Verilog},
author={Nazeih Botros},
year={2006},
publisher={Da Vinci Engineering Press}
}

@misc{arm, title={ARM in the World of FPGA-Based Prototyping}, url={https://community.arm.com/processors/b/blog/posts/arm-in-the-world-of-fpga-based-prototyping}, journal={Arm Community},
year={2016}}

@book{microblaze,
title={MicroBlaze 
Processor Reference 
Guide},
journal={Xilinx},
year={2017}
} 

\end{filecontents}

%s comments
\usepackage{verbatim}

%inline graphs
\usepackage{wrapfig}
% multiple figures on line
\usepackage{subfig}

\usepackage{graphicx}
\graphicspath{ {img/} }

% Caption font size
% https://tex.stackexchange.com/questions/86120/font-size-of-figure-caption-header
\usepackage[font=scriptsize,labelfont=bf]{caption}

%\setlength{\belowcaptionskip}{-10pt}
%\setlength{\abovecaptionskip}{-5pt} % Chosen fairly arbitrarily


\usepackage{fancyhdr}
\pagestyle{fancy}
\lhead{\rightmark}
\chead{}
\rhead{FPGA-based RISC Microprocessor and Compiler (Rev. \dlatestv{})}
\lfoot{Ben Lancaster 10424877}
\cfoot{}
\rfoot{Page \thepage}

\renewcommand{\subsectionmark}[1]{\markright{\thesubsection\ #1}}


\begin{document}

\makeatletter
\DeclareRobustCommand*{\nameref}{%
\color{blue}%
        \@ifstar\T@nameref\T@nameref
        }%
\makeatother

\begin{titlepage}
\begin{center}

\vspace*{5cm}
\Large
\textbf{
%%PRCO304 - Project Initiation Document
%Highlight Reports
FPGA-based RISC Microprocessor and Compiler (Rev. \dlatestv{})
}

\vspace{0.4cm}
\large
%%Space optimised FPGA-based side-microprocessor.
PRCO304 - Final Stage Computing Project
%%EMBEDDED CPU - FPGA-based RISC microprocessor

\vspace{4cm}
\textbf{Ben Lancaster 10424877}\\
\today 


\end{center}

\end{titlepage}

\pagestyle{main}

\section*{Revision History}
\begin{table}[h]
\def\arraystretch{1.5}%  1 is the default, change whatever you need
    \begin{tabularx}{\textwidth}{|l|l|X|}
    \hline
    Date & Version & Changes \\
	\specialrule{2pt}{-2pt}{0pt}
	11/03/2018 & 1 & Initial section outline. \\ \hline
    \end{tabularx}
    \caption{Document revisions.}
\end{table}
\newpage


\section*{Abstract}
ben

\newpage
\renewcommand*\contentsname{Table of Contents}
\tableofcontents
\newpage

\section{Introduction}

\section{Background}
\subsection{Current Implementations}

\section{Objectives}

\section{Legal and Ethical Considerations}
The \scname{} processor will be able to read and write to all data passing through it and in connecting peripherals. 

The processor does not track or store usage behaviour, instructions and their frequency,   memory contents, or timing statistics.

The \scname{} processor is not designed to run general purpose operating systems, such as Linux or embedded RTOS systems. The processor lacks common components required to run modern operating systems, such as a memory management unit (MMU), 

It is not designed to run in high-reliability or safety-critical environments that require established safety standards, such as the UK Defence Standard 00-56 \citep{defstan_0056_2} and IEC 61508 \citep{iec61508}.

This project uses only 1 external library for the processor core's universal asynchronous receiver-transmitter (UART) module that does not depend on any other libraries. The UART module does feature a large first-in-first-out (FIFO) buffer for temporary storage of in- and out- going messages. This FIFO is internal to the FPGA design and so is protected from external viewing or modification.

The compiler sub-project does not use any external library dependencies, does not record telemetry or usage statistics, and does not require an internet connection to run.


\section{Project Management}

\section{Requirements}

\section{High Level Design}

\section{Testing and Verification}

\section{Conclusion}

\section{Appendices}
\subsection{Appendix A. PRCO Core Reference Guide}
\subsection{Appendix B. PRCO Compiler Reference Guide}
\subsection{Appendix C. Project Initiation Document}

\newpage
\bibliography{prco304_h4} 
\end{document}